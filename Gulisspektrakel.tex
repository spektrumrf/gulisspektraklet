\documentclass{spektraklet}

\begin{document}

%----------------------------------------------------------------------
%	arguments:
%		#1	publication number (eg. 1/2015, 2/2015, 3/2015 etc.),
%			the year is filled in programatically.
%
%		#2	the width of the cover image
%
%		#3	the path to the cover image
%
%		#4	optional argument with offsets for the cover image
%----------------------------------------------------------------------
\titlepage{Gulisnummer}{\paperwidth}{bilder/parm.jpg}[0, -2cm]
%\titlepage{1}{0.75\linewidth}{images/dog.jpg}



%----------------------------------------------------------------------
%	There are a couple of fields in the content page that is changable,
%	and some of them are mandatory. The default values are defined in
%	the 'spektraklet.cls' file.
%
%	Mandator:
%		\ChiefEditor{name}
%		\ManagingEditor{name}
%		\Editors{name1\\name2\\name3}	the last name shouldn't be followed by a \\
%
%	Optional:							Default value:
%		\PublicationInformation[text]		Àr ett språkrör för de som studerar - eller låter bli
%											att studera - matematik, fysik, kemi eller datavetenskap
%											på svenska vid Helsingfors universitet
%		\PublicationSupport[text]			Spektraklet får HUS-stöd för föreningstidningar
%		\PublisherName[text]				Spektrum rf
		\PublisherAddress[Exactum]				%Kemiska institutionens svenska avdelning
		\PublisherPostalOfficeBox[PB 68]		%PB 55
%		\PublisherPostalCode[text]			00014 Helsingfors universitet
%----------------------------------------------------------------------


\ChiefEditor{Sebastian Holm}
\ManagingEditor{Daniel Holmberg}
\Editors{
Leo Kolev\\
Walter Grönholm\\
Hugo Åström
}
\CoverPageAuthor{Christina Lassheikki}


% Include the content page.
\contentpage




\begin{ledaren}{}

% Kollage på redaktionsmedlemmarna
\wrappicture{bilder/redaktions_kollage.png}[7 cm][R]


Hej gulis och välkommen till den spektakulära föreningen Spektrum! Vi på redaktionen och hela föreningen önskar dig varmt välkommen till Gumtäkts campus!

Hösten är på kommande och med den inleds en ny hösttermin vid Helsingfors universitet. En hösttermin med nya utmaningar och möjligheter. Med ett läsår kantat av distansstudier och ZOOM-möten i bagaget återvänder fjolisar och äldre studerande till universitetet. För er gulisar innebär hösten en ny, intressant och kanske spännande vardag. En vardag fylld med tentläsning, sitzar (?) och allt annat som hör studielivet till. I vilken utsträckning studier (och studielivet) kommer att påverkas av den rådande pandemin detta år är ovisst. Under hösten kommer en kombination av när- och distansstudier implementeras vid fakulteten i Gumtäkt, med visst förbehåll för förändringar i pandemiutvecklingen. 

I detta gulisnummer av Spektrums blogg/tidning Spektraklet har vi på redaktionen plockat ihop lite nyttig information angående studierna, föreningar och Gumtäkts campus. Vi slängde också in några eventuellt inspirerande artiklar. Här hittas även en presentation av Spektrums medlemmar, vilken kanske kommer speciellt väl till nytta denna höst. Ifall frågor uppstår eller något känns oklart så tveka inte att fråga tutorer, vända dig till äldre spektrumiter eller bläddra igenom denna tidning.

Spektrum är ämnesföreningen för alla svenskspråkiga studerande vid campuset i Gumtäkt. Detta bidrar till en mångfacetterad skara medlemmar och en studiemiljö som överskrider ämnesgränserna. På redaktionen återspeglas Spektrums mångfald i variationen hos våra blogginlägg och kortare artiklar i Spektraklet (\url{http://spektrum.fi/spektraklet/}). Där bjuder redaktionen på bland annat antologier om kvantdatorer, matematiska epos och isbjörnar som koldioxidmått. Som redaktör får man skriva om nästan vad som helst och den som är intresserad får fritt komma med. En ypperlig plattform att släppa fram sin inre Marcus Rosenlund eller Emma Frans.

Från redaktionen och hela Spektrum önskar vi dig ett fantastiskt gulisår!
\end{ledaren}

\begin{ordforandespalten}{Sara Hagström}

\wrappicture{bilder/ordforandespalten.jpg}[4.5 cm][R]

Hej på dig gulis!

Sommaren börjar närma sig sitt slut och i och med detta närmar sig även början på nästa kapitel i ditt liv, nämligen studietiden! Att börja studera på universitet är ett enormt steg i livet och ifall det känns som att allt är nytt och skrämmande behöver du inte oroa dig, så känner de allra flesta i början. Den känslan kommer snabbt att försvinna då du kommer igång med studierna och lär dig känna dina nya studiekamrater!

Att börja studera på universitet erbjuder dig en frihet som du kanske inte har upplevt förr. Du får välja kurser som känns intressanta och skapa en studiehelhet som passar just dig. Kurserna kommer att bestå av bland annat föreläsningar, grupparbeten och självstudier, men i flera fall får du helt själv välja studiestilen som passar dig bäst. Till campus lönar det sig att komma förutom för att delta på kurstillfällen, även för att studera och umgås med andra studeranden. 

Nämligen även om studierna är viktiga, ska du inte glömma bort den sociala aspekten av studielivet. Det kommer att ordnas evenemang av alla sorter, bland annat akademiska fester (dvs. sitzer), filmkvällar, idrottsevenemang, spelkvällar och mycket mer. Det kan ofta i början kännas som att man knappt hinner delta i alla evenemang som man skulle vilja, så du kommer knappast ha brist på något att göra. Jag rekommenderar därmed varmt att redan från första början hänga med i olika gulisevenemang, eftersom du där lär dig känna dina studiekamrater som i bästa fall kan bli vänner för livet!

Slutligen önskar jag dig ännu varmt välkommen till universitetet! Trots det pågående pandemiläget hoppas jag att din studiestart blir så bra som möjligt och hoppas att vi ses i höst på Spektrums evenemang!


\end{ordforandespalten}

\begin{artikel}{Gulisens guide till Uni}{Susanna m.fl.}

\textit{När man inleder sina studier på Uni blir man ofta bombarderad med info från höger och vänster. För att göra saker lite enklare bjuder vi på en lista på några viktiga saker att komma ihåg.}

\textbf{Individuell studieplan}

Den individuella studieplanen, dvs. ISP (eller HOPS som det heter på finska), är en studieplan för kandidatexamen som ska göras upp (helst) under första studieåret. För magisterskedet görs en skild studieplan när det blir aktuellt.

Man gör upp en plan med vilka kurser man planerar att gå och när man planerar att gå dem, och efteråt godkänner ens ISP-handledare planen. Detta är som sagt en plan, och väldigt få följer planen till punkt och pricka, så det är inget man behöver ta stress över. Få den undanstökad i början av studierna så att den inte lämnar och spöka i slutet av kandidatstudierna! På vissa institutioner fixar man ISP:n själv via ett webverktyg och på vissa ordnas någon form av handledning. T.ex. på matten fixas den i samband med handledartutoreringen. Närmare anvisningar finns på avdelningarnas egna hemsidor och i SISU - verktyget, varifrån man anmäler sig till kurserna.

\textbf{Stöd och boende}

Som studerande har du förstås rätt till allmän bostadsstöd samt studiestöd, som består av studiepenning och statsgaranti för studielån. Mer information om detta kan du hitta på FPA:s hemsidor. Kom även ihåg att hålla ett öga på dina inkomster! Det är alltid lättare att låta bli att lyfta någon enstaka stödmånad än att vara tvungen att betala tillbaka eftersom du förtjänat för mycket! Studiestödet kräver också att man uträttar studier på heltid, dvs. man måste få ihop tillräckligt med studiepoäng för att kunna lyfta studiestöd.

Bostad har du förhoppningsvis redan, möjligen även en studiebostad om du har tur. Trots det, så kan det löna sig att söka efter andra alternativ! Studiebostäderna hos HOAS (Helsingin seudun opiskelija-asunnot) är väldigt billiga, men det är en rejäl rulett angående läge och kvalitet. Därför kan det löna sig att söka sig till nationsbostäder, eller till bostäder ägda av SSBS (Svenska Studenters Bostadsstiftelse). De är oftast väldigt billiga, speciellt med tanke på vad “vanliga” bostäder kostar på samma områden.

\newpage

\textbf{Kurser i digitala kunskaper}

Studentens digitalkompetens (2 + 1 sp) är ett bevis på att man klarar av att använda universitetets datasystem, såsom att skriva textdokument, använda bibliotekets nättjänster osv. Kurserna uträttas genom tent och diverse andra uppgifter beroende på ens huvudämne. Allt material finns på nätet.

Dessa kurser kan vara ett bra sätt att bekanta sig med allt vad universitetets datasystem har att erbjuda, och studierna i detta kräver inte mycket tid. Det kan räcka med en eftermiddag eller två, beroende lite på ens utgångskunskaper och hur noggrant man läser igenom materialet. 

\textbf{Obligatoriska språkstudier}

Till kandidatexamen är det obligatoriskt att utföra 4 sp i ett främmande språk (engelska, tyska, franska, spanska, italienska eller ryska) och 3 sp i det andra inhemska språket (finska eller svenska). CEFR-utgångsnivån skall vara B2 för engelska och B1 för alla andra språk. Språkstudierna kan uträttas som tent eller som kurs. 
Om man lyckades få någorlunda okej vitsord i de språk man skrev i studentexamen så lönar det sig att försöka tenta bort språkstudierna. Får man inte godkänt i tenten så kan man alltid gå kursen. Språkstudierna rekommenderas att utföra under andra studieåret, men det kan också löna sig att fråga om man kan gå tenterna redan som gulis, då gymnasiets språkstudier ännu är färska i minnet.

Tenten i det främmande språket består av en skriftlig och en muntlig del som är värda 2 sp var. Får man godkänt i ena delen men inte den andra så räcker det att gå en 2 sp kurs, annars får man gå en 4 sp kurs. Man tentar de olika delarna vid skilda tillfällen och den muntliga delen består av en läsförståelse samt diskussion. Man kallas till diskussion endast om man fick godkänt i den skriftliga delen. Kurser och tenter i engelska
ordnas i Gumtäkt, och de är möjligen institutionsspecifika. Detta meddelas ofta i kurs- eller tentbeskrivningen.

Det andra inhemska språket är endera finska eller svenska. Råkar det sig så att ditt
modersmål är finska men du skrev studenten på svenska eller, vad gäller IB studerande, om du gick högstadiet på svenska så rekommenderas att du dubbelkollar t.ex. vid din institutions kansli vilket språk du borde studera, för chanserna är stora att du borde studera finska som andra inhemska språket fastän det är ditt modersmål. Orsaken varför det kan bli lite rörigt vad gäller språkstudier, är att det i instruktionerna ofta står modersmål när dom egentligen menar skolspråk. Så kallade modersmålsstudier, som möjligen inte är i ditt officiella modersmål, utförs i samband med kandidatexamen. 
Finska-tenten består oftast av en eller ett par essäer. Får man godkänt i detta så kallas man till en kort diskussion. Tenten är alltså inte så speciellt omfattande och om ens finska är någorlunda bra, så lönar det sig att försöka tenta den. Ifall du är tvåspråkig, så kommer tenten antagligen kännas väldigt enkel. Engelskan är däremot definitivt krångligare!


\textbf{Andra ämnesstudier}

Andra ämnesstudier, eller biämnesstudier i folkmun, kan redan påbörjas under första
studieåret, men det lönar sig att komma ihåg att vissa ämnen utanför vår fakultet är begränsade, t.ex. psykologistudier kräver inträdesprov. Information om möjliga begränsningar hittas ofta på institutionernas egna hemsidor. Ämnen från vår fakultet kan studeras fritt. Kraven för hurdana biämnen som kan medräknas i kandidatexamen m.m, kan man hitta i sin egen institutions studieguide, men i allmänhet krävs det att man har åtminstone en registrerad biämneshelhet på 25-35 studiepoäng.

\picture[]{bilder/bibba.jpg}

\newpage 

\textbf{Kampusbiblioteken och kursböcker}

Det finns kampusbibliotek i Centrum, Mejlans, Vik och Gumtäkt. Det som vi naturvetare har mest användning av är (förstås) Gumtäkts kampusbibliotek som finns i Physicum. Studiekortet kan aktiveras som bibliotekskort vid alla kampusbibliotek.

Det som ni främst kommer att ha användning av, som första årets studeranden, är kursböckerna som finns att låna. Kursböcker kan vara mycket dyra och detta kan vara ett bra sätt att spara pengar. Men ta i beaktan att många tänker på detta sätt så det lönar sig att vara ute i god tid, annars är alla böcker redan utlånade. Ett annat alternativ kan vara att kolla med äldre spektrumiter om de har kursböcker att låna ut eller sälja billigt. 
Gumtäkts kampusbibliotek har även en läsesal där det finns upplagor av (de flesta) kursböcker, så i nödfall kan man använda sig av dem. 

Om man har användning av boken i flera kurser eller om det är en bra ”grundbok” som man kan använda som referensmaterial senare i studierna, så kan det möjligen löna sig att köpa den. Databöcker lönar det sig (nästan) aldrig att köpa eftersom de ofta är väldigt dyra, föråldras snabbt och oftast kan man hitta allt material man behöver på nätet.

\textbf{Unisport}

Tycker du om att träna rekommenderas definitivt att du betalar Unisports träningsavgift. Med den avgiften du betalar för ett år tränar du 2-3 månader på de flesta kommersiella gym, så man sparar massor med pengar. För summan får man besöka alla Unisports gym och ta del i de träningspass som ordnas. Man får även rabatt på kurser som ordnas och olika tjänster som erbjuds, såsom anlitning av en personlig tränare. 

Första gången man betalar träningsavgiften måste man göra det på nåt av Unisports gym. Efter detta kan man betala träningsavgiften via Unisports hemsidor, om man så vill.
 
Det är ganska populärt att träna på Unisport, så träningspassen blir snabbt fulla. Det lönar sig att endera boka i god tid eller att hålla utkik ett par timmar innan passet börjar, så man kan knycka en plats som någon avbokat. Man måste avboka ett pass minst en timme innan passet börjat, och glömmer man att avboka passet så är man tvungen att betala böter. Man kan inte anmäla sig till nya pass innan böterna är betalda. 

Det lönar sig att använda gymmet under opopulära tider (mitt på dagen eller strax före stängningsdags), eftersom det efter fyra brukar vara helt fullpackat på gymmet och man kan vara tvungen att köa till maskiner.

\end{artikel}

\begin{artikel}{Presentation av styrelsen}{}
\begin{twocolumns}

\picture[]{bilder/stysse_ordf.png}[\textbf{Bowsara} \\ \\ Sara Hagström, Ordförande \\ \\ 4. årets mattastuderande \\ \\ Spektrums final boss som kämpar hårt för att kontrollera kungadömet.]

\columnbreak

\picture[]{bilder/stysse_vice.png}[\textbf{Bowseva Jr.} \\ \\ Eva Ahlgren, Vice ordförande \\ \\ 4. årets kemistuderande \\ \\ Möjligen den framtida tronarvingen som hjälper till härskaren Bowser med sina viktiga uppgifter och tar emot  olika uppdrag för att utöka sin kunskap.]

\columnbreak

\picture[]{bilder/stysse_sekr.png}[\textbf{Walle} \\ \\ Kalle Kupi, Sekreterare \\ \\ 4. årets fysikstuderande \\ \\ Skriver ner wad som yttras under warje möte. War på er wakt, era ord kan anwändas mot er.]

\columnbreak

\picture[]{bilder/stysse_skat.png}[\textbf{Yoshmil} \\ \\ Emil Amnell, Skattmästare \\ \\ 4. årets matematikstuderande \\ \\ Slukar i sig kvitton, betalningar och pappersarbete. Spottar ut budgetar och finanser för Spektrums evenemang och inköp.]

\columnbreak

\picture[]{bilder/stysse_prog.png}[\textbf{Mariella} \\ \\ Gabriella Karhulahti, Programchef \\ \\ 3. årets fysikstuderande \\ \\ Hittar alltid på nya äventyr åt spektrumiterna med hjälp av programkommittén. Dessa kan vara storslagna partyn eller olika utmaningar.]

\columnbreak

\picture[]{bilder/stysse_stud.png}[\textbf{Antolina} \\ \\ Anton Taleiko, Studiesekreterare \\ \\ 3. årets datavetare \\ \\ Sköter intergalaktisk kommunikation med aktörer utanför föreningen och välkomnar nya stjärnor till universitetets universum med sina tutorer.]

\columnbreak

\picture[]{bilder/stysse_milj.png}[\textbf{Tirkkoigi} \\ \\ Andreas Tirkkonen, Miljö- \& kulturansvarig \\ \\ 4. årets fysikstuderande \\ \\ Spektrumrikets lojale beskyddare. Hjälper gärna till att lösa regimens kriser tillsammans med den övriga regeringen.]

\columnbreak

\picture[]{bilder/stysse_jaml.png}[\textbf{Nozeach} \\ \\ Chloé Nozais, Jämställdhetsansvarig \\ \\ 4. årets fysikstuderande \\ \\ Sätter andra före sig själv i hopp om att alla ska ha det bra och behandlas väl.]

\end{twocolumns}
\end{artikel}

\begin{artikel}{Pseudointellektuellt svammel}{Jere}

\textit{År 1 735 publicerade den svenske vetenskapsmannen Carl Linnaeus (observera att han adlades först år 1 757\footnote{Blunt, W. Linnaeus: The Compleat Naturalist, Frances Lincoln Ltd., 2004, s. 1 71}) den första upplagan av kategoriseringsverket Systema
Naturae. Han var den första att gruppera människorna och aporna i samma släkte. Intressant nog skiljer han inte åt oss från våra kusiner hankeiterna utgående från biologiska olikheter, utan med filosofiska aforismen nosce te ipsum, känn dig själv. Han menar att självkännedom är det definierande draget för oss som en art\footnote{Klein, R.A. Sociality as the Human Condition: Anthropology in Economic, Philosophical and Theological Perspective, Koninklijke Brill NV, 2011 , s. 59}.}

%\wrappicture{bilder/matrix.jpeg}{}{}{Begreppet ”känn dig själv” förekommer i många variationer, t.ex. i filmen The Matrix (1999).} %Caption breaks the image alignment

\begin{wrapfigure}{L}{6cm}
	\includegraphics[width=6cm]{bilder/matrix.jpeg}
	\caption*{Begreppet ”känn dig själv” förekommer i många variationer, t.ex. i filmen The Matrix (1999).}
\end{wrapfigure}

Aforismen ifråga är dock äldre än de knappa trehundra år emellan oss och upplysningens Sverige. Det är uppenbart att aforismen var viktig även för antikens greker, då begreppet var inskrivet i väggen vid Apollons tempel i Delfi\footnote{Miller, J. Examined Lives: From Socrates to Nietzsche, 1. t., Farrar, Straus and Giroux 2011, s. 22}.

Den skarpa läsaren frågar sig varför detta är relevant för ett gulisnummer. Studenttidningar och dylika infoblad brukar innehålla en hel del goda råd åt de nya. Skaffa bostad, kom ihåg att delta i fritidsverksamheten, gnäll om pengar av FPA osv. Bland allt detta kan det vara svårt att komma med något revolutionerande och intressant. En svår målgrupp för skribenter, vilket i första hand beror på att de naturvetenskapliga gulisarna tenderar att vara oskulda. Alltså vad gäller studielivet. Det är krävande att förklara hur allt fungerar, lite som att förklara åt ordningsvakten att nej, det var varken du eller din kompis som spydde i hörnsoffan. Kemister destillerar olika saker med varierande framgång och här följer mitt försök att destillera en gnutta sanning.

Sun Zi, en kinesisk general född ca 500 f.Kr., skrev klassikern Krigskonsten. I detta epos konstaterar han att ifall man känner sig själv och fienden, behöver man inte frukta resultatet av hundra strider\footnote{Sawyer, R.D. The Seven Military Classics of Ancient China, Basic Books, 2007, s. 421 – 422}. Men vem är fienden? Det kan vara FPA med deras inkomstgränser från helvetets åttonde krets eller kanske ens egen lathet inför tenten. Faktum är att motståndarna är många och av varierande form, men självkännedom om ens styrkor och svagheter är konstant.

Och här kommer vi tillbaka till att känna sig själv, då det kan vara svårt att veta vad man vill. Studielivet är dock en utmärkt chans att vidga sina vyer. Många lär sig sina fysiska gränser på festernas sena timmar. Andra lär sig mentala dygder, såsom tålamod, genom sin första styrelsepost (det bör påpekas att det som händer i styrelsen stannar i styrelsen). Detta för att nämna några exempel.

Risken finns att självkännedom är ett fenomen av zenbuddistisk kaliber. Man kan inte bli upplyst om man ämnar bli upplyst, utan man måste snubbla på sanningen av misstag. Inte så långt ifrån att snubbla på tamburmattan i morgonmörkret.

Men hur borde man gå till väga? Det finns knappast ett definitivt svar. Mitt råd lyder att pröva olika saker, även om man tvekar, t.ex. tidigare nämnda styrelseposten. Av erfarenhet är det känt att ofta blir man positivt överraskad. Olika upplevelser avslöjar nya sidor av ens karaktär. Besök även andra föreningar, där man med god sannolikhet lär känna färggranna personer. Ofta har de annan inställning än en själv till flera frågor, vilket i bästa fall leder till ögonöppnande aha! -upplevelser. Även Paul Bragiel, VD för i/o Ventures, ett av de största investeringsbolagen i Silicon Valley, konstaterade att det bästa med universitet är att träffa spännande personer.

Spektrum är en liten men hårt sammansvetsad förening, vilket gör det enkelt att lära känna de andra. I mitt tycke är det också ett utmärkt sätt att utmana sig själv och knyta nya vänskapsband. Alla är välkomna att delta i verksamheten, även om de inte ämnar fortsätta studera naturvetenskaper. Några av er kommer att hitta er plats i livet på något annat ställe. För er vill jag citera Marcus Aurelius (1 21 – 1 80 e.Kr.), romersk kejsare och en av de få män som uppfyller Platons utopistiska vision om en filosof som kung\footnote{Aurelius, M. Meditations: A New Translation, övers. Hays, G., Modern Library, 2002, s. i}: 

Även om du gett upp hoppet om att bli en stor tänkare eller vetenskapsman, ge inte upp hoppet om att uppnå frihet (7.67)

\end{artikel}

\begin{artikel}{Medlemspresentation}{}

\textit{Eftersom det är mycket nytt när man är gulis, så tänkte vi att det kunde vara bra med en presentation av aktiva medlemmar som på ett eller annat sätt utmärker sig. Detta är alltså inte alla Spektrums medlemmar! Vi bad helt enkelt Spektrums medlemmar att skicka in en kort textsnutt om andra medlemmar och här är nu en sammanställning av dem. Förhoppningsvis gör presentationerna det lättare att lära känna och känna igen spektrumiterna. Om inte, så känns de igen på att de i princip är de enda som talar svenska uppe i Gumtäkt...}

\begin{twocolumns}

% Man är antingen andra årets kemist/fysiker/whatever eller sen nämner man inte månte året (gamyler kanske exception). Styssemedlemmar samt idrotts-, kafferums- och ADB-ansvarigas poster bör nämnas. PRK, värdar etc. behöver inte nämnas. Då sparar man uppdatering och huvudvärk :)

% Om ingen i redaktionen känner individen skall man fråga sig om individen i fråga har varit någo aktiv. Om jo, fråga andra spektrumiter för att hjälpa till med presentationen.

% Håll presentationerna ett extra år (RGL medlemmar två år) efter att hen har blivit totalt inaktiv. Ibland kommer de också tillbaka! Tillsätt en "ta bort 20XX" kommentar på sådana

\textbf{Alfredo} - Skön typ med ännu skönare humor. Fäktare, geolog och f.d. tutor.

\textbf{Anders} - Aktiv ekonometriker som kan höras försöka påbörja Bang! i kafferummet. Alternativt hittas han ofta på Unicafe, var han äter i princip alla dagens måltider. % ta bort 2022?

\textbf{Andreas B} – Denna andra årets fysiker är en charmig och karismatisk ung gosse som behandlar utmaningar med nerver av stål. Han övergiver aldrig sitt lugn och medför stabilitet i vänkretsen.

\textbf{Andreas T} - Fysiker med positiv inställning till vad han än tar sig an. Kafferumsansvarig och kulturansvarig.

\textbf{Anna} - Glad kemist som ofta syns till i Kafferummet och på Klubben. Lär vara en mästare på pussel och molekylsimuleringar.

\textbf{Anton} - Datavetartrumpetistesbobo som kan ses spela GTA på discord. Del av soon-to-be-infamous Lärkangäng. Föreningens studiesekreterare.

\textbf{Aslak} - Spektrums egna lapplänning som spenderar misstänksamt mycket tid till sjöss. Studerar fysik, är en av Spektrums trakasseriombudsmän och är grundaren av Spektrums bokklubb. Medlem i suprekordlaget Mys Mys. \emph{RGL}

\textbf{Chloé} - Fysikstuderande och Spektrums jämställdhetsansvariga. Man kan samla plusspoäng med henne om man har en iphoneladdare att låna.

\textbf{Daniel} - Data fysiker hybrid. Spektrum appens och Discord-bottens kodare. Medlem i suprekordlaget Mys Mys. Redaktionschef.

\textbf{Dennis} - Ett geni som har kommit längre i matematikssstudierna än sin tutors tutor.

\textbf{Emil A} - Statistiker och tutor som gärna spelar brädspel och kläcker ur sig några dank memes nu och då i kafferummet. Sköter föreningens ekonomi.

\textbf{Emil L} - Invandrade från Lancaster en dag och har sedan dess ofta setts i kafferummet, våndandes över sitt jobb på acceleratorlabbet. Har migrerat västerut. % ta bort 2022?

\textbf{Emma M} - En utmärkt tredje årets geolog från Sibbo med en vacker talang för sång. När hon inte är en vandrande geolog sjunger hon i Luciatåget på julen. % ta bort 2022?

\textbf{Emma S} - Glad och rolig geograf som once in a blue moon vågar sig till kafferummet. Ses främst på Spektrums sitzer och andra evenemang.

\textbf{Eva} - Kemist som vet hur man festar. Om du känner att du saknar ett par till beer pong eller någon att shotta med så är Eva your girl. Spektrums vice ordförande.

\textbf{Frans} - Fysiker som alltid uppskattar en god whiskey. Även känd som ``Herra Isoherra'', och för att ha varit den officiellt kompaktaste ordföranden inom Spektrum. \emph{RGL} % ta bort 2023?

\textbf{Gabriella} - Born and raised in Replot, Pampas. En ''priima'' fysiker javisst, men var studerar hon? Spektrums programchef.

\textbf{Hanna} - Andraårskemist som har ett öga för mode och dessutom en hemlig musikalisk talang.

\textbf{Harald} – Sportig och aktiv andra årets fysiker som syns till på allt från sportevenemang till sitser. Nuvarande tutor.

\textbf{Henrik} - Stolt pampes och smart kemist. Kallas ibland vid namnet “Stubb”, som emellanåt sjungs entusiastiskt på sitzer och ingen vet riktigt varför... % ta bort 2022?

\textbf{Hugo} - Kemist och kvantdatorspecialist. Spektrums idrottsledare och vår egen Dressmann modell.

\textbf{Jeremias} - Lång matematiker som beslöt sig för att bli doktor i datavetenskap. Klättrar, kastar frisbees och spelar bridge med nylänningar. \emph{RGL} % ta bort 2022?

\textbf{Jim} - En rojsig kemist från Borgå som tacklar folk i fritiden.

\textbf{Julia} - Trevlig och ärlig kemist från Sibbo som tror sig vara mer awkward än vad hon egentligen är. Gillar heart-to-heart diskussioner.

\textbf{Julius} - Andra årets tvex-fysiker och äkta stadibo. Han kan verka lugn på utsidan men är en äkta pajas på insidan.

\textbf{Kalle} - Fysiker, hobbyfilosof, nuvarande sekreterare och f.d. tutor. Gillar musikalen Hamilton, anime och är en aktiv deltagare i Spektrums bokklubb.

\textbf{Leo} - Snäll andra årets geograf med mystiska östeuropeiska rötter.

\textbf{Luukas} - En energisk kemist som är med på alla aktiviteter inkluderande dryckestävling under en sitz. Han tycks också ha en tendens att tappa bort sig under evenemang tex på främmande åländska fester?

\textbf{Mackan} - Spektrums jazz-hipster som hummar eller trummar alltid på nåt. Studerar matematik. En eldsjäl i skapandet av den nya sångboken.

\textbf{Markus} - En hacker och "inside-man" från Sibbo. ADB-ansvarig. Officiell medlem i Spektrums inofficiella go klubb.

\textbf{Meri} - TvEx-kemist och f.d. Sveaborgfånge som alltid uppskattar finska ordvitsar. Har flest Jukola- och Venla-starter i Spektrums namn. % ta bort 2023?

\textbf{Micaela} - Aktiv och glad tredje årets fysiker. Kan verka ofarlig, men låt inte skenet bedra eftersom hon lär kunna en hel del om kampsport. En av två trakasseriombudsmän.

\textbf{Michaela} - Geograf som gillar att hitta på kortspel. Syns alltid då och då på evenemang.

\textbf{Natalie} - Andra årets datavetare från Kyrkslätt. Kan ofta hitttas på Spektrums D\&D-sessioner där hon är känd som fågelnunnan Kyarr.

\textbf{Niko} - En aktiv kemist som njuter av att festa. Honom hittar du garanterat på de flesta evenemang och speciellt sitzer!

\textbf{Olavi} - En riktigt tvexig andra årets fysiker och äkta pokémon master. Gillar kortspel och bilis. En förevig helsingforsare.

\textbf{Oliver} - En äkta fysikergamyl som alltid syns till på Klubben då det händer något. Trösklar är hans överraskande fiende, öl ej! \emph{RGL}

\textbf{Oskar Br} - Matematiker. Syns i kafferummet eller på Spektrums sportevenemang. Ser till att vi deltar i Jukola i år.

\textbf{Oskar F} -  Kapabel matematiker, de flesta känner honom som ``Åland''. Vet mycket om drycker, både som klubbmästare och som förbrukare. Medlem i suprekordlaget Mys Mys.

\textbf{Otto} - En realist från Sibbo som alltid är redo att hjälpa föreningen och får saker gjort. Har den största namnskylten på Klubben. \emph{RGL} % ta bort 2023?

\textbf{Paola} - Humoristisk andra årets geograf och tutor som älskar en bra GT. Testade vattnet inom politiken detta år.

\textbf{Petter} – Andra årets fysiker från Pargas vars favoritfärdmedel är cykel. Han säger inte så mycket men när han gör det har han alltid något vettigt att påstå.

\textbf{Rasmus N} - Även kallad Nisse. Aktiv pünchenhävare som inte behöver mer än en flaska Leijona för att underhålla sig i flera timmar. Köksmästare och kafferumsansvarig.

\textbf{Robert Ho} - Glad och chill datalog. Kan alla de bästa babyskämten. % ta bort 2022?

\textbf{Robert Hä} - Nördig matematiker som flydde till Innsbruck, Österrike för ett år. Fick nog av joddlande och beslöt sig att returnera till Finland. % ta bort 2023?

\textbf{Robert P} - Ex-fysiker som gjorde ett bra beslut och migrerade till datavetanskap. Tennisproffs.

\textbf{Robin} - Trevlig, dataspelande fysikstruderande som härstammar från Veikkola. Spektrums Go master. % ta bort 2022?

\textbf{Ron} - En andra årets fysiker som alltid gärna hjälper till. Expert på att skjuta sneda kaströrelser med Nerf pistoler. 2020 fyssagulisarna fashionista.

\textbf{Sara} - Matematiker som fungerar som föreningens ordförande. Hittas ofta i kafferummet spelande diverse kortspel eller på sitzer i vilka hon \emph{aldrig någonsin} svinar. Spektrums största svamlare.

\textbf{Samuel} - Datavetare som tillbringat andra halvan av sitt gulisår i samhällets tjänst i armén. % ta bort 2022?

\textbf{Sebastian ``Sebbe'' H} - Sportig fysiker från Österbotten. Kan hittas högst uppe i Physicum där han studerar molnbildning. Chefredaktör (som ni redan visste, eller hur?)

\textbf{Simon} - En allt för ärlig matematiklärarstuderande som vet hur man njuter av livet. Golfar med bollar och frisbees. ’Nuff said.

\textbf{Tobias} - Åbokemist i början, nuförtiden huvudstadsbo. Går med stil, oberoende hur hett det är. Kan ses på Klubben lite nu som då. % ta bort 2022?

\textbf{Toffe H} - Tredjeårsfysiker med en killer smile. I hans egna ord: sheltered white boy. Panini 4 life.

\textbf{Toffe F} - Hör till den alltid lika soliga Spektrala Pampasligan. Kan hittas i acclabbet eller i kafferummet. Visar närapå omänskliga kunskaper då någon har problem med datorn. \emph{RGL}

\textbf{Victor} - Svinande(?) fysiker som gör allt enligt sitt eget tempo, dock ändå får saker gjort. Då någon säger ansvar, vänder Victor andra kinden till.

\textbf{Viktor ``Horsmanheimo'' Horsmanheimo} - Datavetare vars närvaro i Exactum och sitzer inte går att missa.

\textbf{Viktoria ``Titti''} - Matematiker, trots att hon nästan valde kort matematik i gymnasiet. Går på sushi och aperol.

\textbf{Waffe} - Gamyl som kodar eget programmeringsspråk varav ni kommer att höra nog. Spelproducent, rollspelare, dungeon master (not the kinky kind) och snabbkaffe-med-salmiak-gillare. Borde skriva på sin gradu.

\textbf{William} - Trevlig kemist och sporadisk tyskastuderande. Syns i kafferummet alltid då och då.

\end{twocolumns}

\picture[]{bilder/rubber_sheet.png}[xkcd.com]
\end{artikel}

\begin{artikel}{Ett litet val i all hast - hur jag blev tvåspråkig}{Meri Sillanpää}[]
\begin{twocolumns}

\textit{Ensimmäinen yliopistopäiväni kolme vuotta sitten. Keskustassa aamulla ruuhkaa ja olen myöhässä vartin. Integraalimäestä hengästyneenä ja hivenen hikisenä saavun vihdoin Chemicumin aulaan. Aula on lähes tyhjä, muutama tuutori istuu pöydän ääressä ja ennen kun he neuvovat minut oikeaan luentosaliin, saan pienen paperilapun täytettäväksi.}
 
\vspace{0.2cm}
\noindent\fbox{%
    \parbox{\linewidth}{%
        \vspace{0.1cm}
        Nimi:
        
        Olen kiinnostunut:
        
        $\square$ Ope-opinnoista
        %\mbox{\ooalign{$\checkmark$\cr\hidewidth$\square$\hidewidth\cr}}
        
        $\square$ Kaksikielisestä tutkinnosta
         \vspace{0.2cm}
    }%
}
\vspace{0.2cm}

Nimi? Helppo homma. Kiinnostuksen kohde oli sen sijaan kiireessä hieman vaikeampi kysymys. Tiesin etten halua opettajaksi, sen sijaan en ollut koskaan kuullutkaan kaksikielisestä tutkinnosta. Muistan edelleen mitä ajattelin seistessäni lappu kädessä keskellä aulaa, myöhässä ensimmäiseltä orientaatio luennolta: ”Voihan siitä olla kiinnostunut, ei se tarkoita että tutkintoa tarvitsisi tehdä kaksikielisesti”. Joten myöhässä ja kiireessä rastitin ruudun ”kaksikielinen tutkinto”, annoin lapun takaisin tuutorille ja kiiruhdin luennolle.

\columnbreak
Utöver detta kommer jag inte ihåg mycket mera av min första dag vid Uni, men jag kommer ihåg detta lilla val som gjordes i all brådska för det har påverkat mycket mitt studieliv. Tack vare den tvåspråkiga examen (TvEx) kom jag också med i ämnesföreningen Spektrum, som är en ypperlig plats för att utforska studielivets bästa sidor; fest, Klubben, sits, nya vänner och förstås billiga drycker. Att öva samt lära sig ett nytt språk lyckades ganska naturligt sådär samtidigt. När man börjar på TvEx-programmet förväntar sig ingen att du skulle kunna det andra språket perfekt, det enda som krävs är motivation och vilja att lära sig. I Spektrum kritiserar man inte språkfel, utan man uppmuntrar andra för att försöka deras bästa. Man skulle kunna säga att det finns en oskriven regel inom Spektrum: Om du vill, så får du alltid prata ditt eget modersmål, oberoende om det är finska eller svenska.

Itselle ehdottomasti paras tapa oppia uutta kieltä oli kuunnella ja käyttää sitä. Olkaa siis itse aktiivisia, ennakkoluulottomia ja osallistukaa myös vieraskielisiin tapahtumiin. Mutta ennen kaikkea, nauttikaa kaikista niistä kokemuksista, joita tuleva (mahdollisesti kaksikielinen) vuosi voi teille tarjota.

\end{twocolumns}
\end{artikel}

\newpage

\begin{artikel}{Presentation av ämnesföreningar}{}

\textit{Som kanske inte så många av er vet så finns det fler ämnesföreningar inom Gumtäkts
väggar. Spektrum är den enda helt svenskspråkiga, men finskspråkiga finns fler än väntat. Så
om man vill finslipa det andra inhemska språket lite så talar de finskspråkiga studerandena
mer än gärna med en. Här har en del av de finskspråkiga föreningarna skrivit en liten
presentation om deras förening, på svenska! Ta er en titt, så finner ni kanske en ny
vänförening.}

\textbf{Limes}

Limes är en ämnesförening från Gumtäkt som grundades redan 1936.
Om du studerar vad som helst i Gumtäkt är Limes din förening. ;)
Vi organiserar olika evenemang - från bastukvällar och exkursioner till sitsar och bileen.
För oss är det största evenemanget Limeksen Appro, som årligen får 2000-3000 deltagare.
Vi säljer också läroböcker och halarmärken förmånligt.
Välkommen att bli medlem vid vårt kontor som ligger i Exactum, rummet C132! <3
(Dvs. brevid kafferummet - Redaktionen)

\textbf{HYK}

Helsingin Yliopiston Kemistit ry eller HYK är den finskpråkiga ämnesföreningen för kemister i Gumtäckt.
HYK är grundad år 1927 och är en av de äldsta studentförenringar i Helsingfors
Universitet.
Vi älskar traditioner men också att oganisera nya händelser för våra medlemmar.
Man hittar kemister i svarta overaller i Opsos, HYKs kafferum, som ligger i Chemicum.
Välkommen!

\textbf{Geysir rf}

Geysir rf grundades år 1997 och är blivande geofysikers intresseorganisation. Vårt mål är att ordna en så mångsidig underhållning som möjligt för våra medlemmar: exkursioner,
dvs studiebesök hos studie- och forskningsanstalter samt på isbrytare och på havsforskningsfarty, idrott (bl. a. väggklättring), kultur (filmer, teater, konserter) samt soaréer (film- och spelkvällar). Exkursionerna till utlandet ingår sedan länge i Geysirs traditioner. Sådana besök inträffar vanligtvis med två års mellanrum. Vi har nyss varit på Island, vid Spetsbergen, på Azorerna, på Nya Zeeland och i Etiopien. Målet för resan år 2014 var Ungern, och var och en är välkommen att komma med idéer till vår nästa resa för år 2016!

\textbf{Synop ry}

Synop ry är meteorologiska ämnesorganisationen som har hjälpt människor rikta sina
ögon mot moln för redan 45 år. Vi är ganska liten, men aktiv organisation som ordnar olika
slags evenemang från sport och filmkvällar till bingor. Den största bingon ordnas varje vapp
och alla är välkomna. Mer information om oss kan ni hitta från vår webbsida www.atm.helsinki.fi/synop eller studentrumet i Physicum där det ofta finns några medlemmar.

\textbf{Meridian rf}
Tycker du om att vara vaken helan natten och att sova hela dagen? Meridian rf
(Meridiaani) är en ämnesförening för astronomistudenter i Helsingfors universitet. På hösten
har vi olika slags evenemang för unga astronomer, vi grillar, spelar brädspel och vår ölklubb träffas varje månad. På våren organiserar vi ett berömd rymdparti Yuri's Night. För att observera stjärnor, planeter och galaxer användar vi en 60-cm teleskop i Metsähovi
Obsevatoriet i Kyrkslätt. Du kan hitta oss i Gumtäkt i Physicum's Studentrummet OH med de
andra fysikerorganisationerna.

\end{artikel}

\begin{artikel}{Byggnader i Gumtäkt}{Sandra m. fl.}

\textit{Hej alla nya studeranden och även ni lite äldre. Här kommer en liten presentation över byggnaderna som finns vid Gumtäkts campus. Så om ni aldrig har varit där tidigare,
eller inte tillbringat tid där på så länge att ni inte minns hur det ser ut mera, så kan
det vara en bra idé att ta er en titt på denna artikel.}

\picture[width=12cm]{bilder/chemicum.jpg}

\textbf{Chemicum}

Kemiströvarnas härbärge. Byggnaden är åtskild från de övriga, vilket brukar resultera i att endast kemister rör sig i ”Chemen”. Om vi andra studeranden drar oss ditåt är det för det mesta för att stilla hungern i matsalen. Svenska chemen fanns tidigare i denna byggnad, men tyvärr har vi fått säga adjö till detta svenskspråkiga fenomen. Spektrums kafferum brukade vara i Chemicum, förrän alla matematiker började gnälla om den långa vägen dit.

\picture[width=7.5cm]{bilder/exaktum.jpg}

\textbf{Exactum}

Det är här alla matematikrumpnissar och datalogvildvittror håller hus. Den ökända Integralbacken (d.v.s. Djävulsbacken) leder upp till denna byggnad, och är orsaken för flera förseningar och diverse mindre skador. Datalogerna befinner sig oftasts på andra våningen eller i källarvåningen, medan matematikerna håller hus på tredje våningen. På tredje våningen finns även en mängd allmänna brädspel, som schack och go. I Exactums första våning ligger både Linus Torvalds auditorium och Lars Ahlfors auditorium, där de flesta tenterna skrivs. Utöver detta finns Spektrums kafferum i Exaktum, där alla flitiga och mindre flitiga spektrumiter håller hus.

\picture[width=12cm]{bilder/physicum.jpg}

\textbf{Physicum}

Här styr och ställer grådvärgsfysikerna. Ska man till Physicum brukar oftast den högra delen av djävulsbacken vara den rätta vägen att ta. I Physicum finns det både det ena och det andra. Biblioteket, där man förvånansvärt nog kan låna böcker eller bara sitta och studera ostört, samt ett café, där det säljs paninin, baguetter (eller ``patonkin'', som de i folkmun också kallas) och goda bullar. ”Sandlådan” är också ett ställe där flitiga studeranden kan samlas; denna del finns på högsta våningen i Physicum. Det finns självfallet också utrymme för klassrum. I dessa klassrum finns både fysiker, geologer och andra fysikintresserade människor.

Från och med 2018 delades fysikavdelningen vid Helsingfors Universitet in i två delar: en för det mesta oförändrad, “allmän” fysikavdelning och till Institute for Atmospheric and Earth System Research (INAR), som är en forsknings- och undervisningsavdelning inom atmosfärvetenskap. I Physicum hittas även HIP- Helsinki Institute of Physics.

På Physicums andra våning finns även kansliet, dit ni borde gå för att bland annat få ett fint klistermärke på ert studiekort. Om man har andra frågor är det också dit man borde röra sig. 

\picture[width=12cm]{bilder/acclabbet.jpg}

\textbf{Acceleratorlabbet}

Acceleratorlabbet är stället alla viktiga fysiker samlas på. Vem de är och vad de gör där vet ingen, men rykten går kring Gumtäkt som en löpeld.
\vspace{2cm}

\textbf{Övriga byggnader}

Förutom dessa finns Dynamicum, där de icke-Astrid Lindgren-relaterade meteorologiforskarna befinner sig; samt Unisport, dit alla sportiga och ibland icke-sportiga beger sig för att röra på sig. Spektrum ordnar ofta diverse gemensamma idrottstillfällen här. I våras inleddes byggandet av Helsingfors naturvetenskapliga gymnasiums nya byggnad vid vårt campus. Planeras stå klart till 2023, vilket innebär ett par terminer med en liten byggnadsplats mellan Physicum och Exactum. 


\picture[width=12cm]{bilder/unisport_dyn.jpg}

\picture[width=12cm]{bilder/bygget.jpg}

\end{artikel}

\begin{artikel}{Haikun: Veckoslut}{}
En diktsamling av pseudo-moderna haikun av den snart världsberömda, dock alltid söndriga, yrbollspoeten Emil L.
\vspace{0.5cm}

\textbf{Förord}

Min kära Jonas.\\
Du sade: EN artikel.\\
Min förrädare.


\vspace{0.8cm}

\textbf{Bok 1: Att vaka}

\begin{flushleft}
Skrivit på kandin.\\
Hela dagen i Gumtäkt.\\
Dags att gå sitsa.
\end{flushleft}

\begin{center}
Alla sätter sig.\\
Nu äter vi och sjunger;\\
med snaps, efter snaps.
\end{center}

\begin{flushright}
Helan och halvan,\\
och någonting om minnet.\\
Nej! Int Theobald...
\end{flushright}

\begin{flushleft}
Nu kommer punschen.\\
Usch vad jag hatar punschen.\\
Men dricker ändå.
\end{flushleft}

\begin{center}
Nån spelar beerpong.\\
Hen kastar bollen, missar.\\
Bollen hittas ej.
\end{center}

\begin{flushright}
Sen sade Herr J:\\
``Ditt hår luktar till anus''\\
Gjorde mig ordlös.
\end{flushright}

\begin{flushleft}
Herr V drack en öl.\\
Han skrattade, men varför?\\
Han är i fyllan.
\end{flushleft}

\begin{center}
En shot, två shot, tre.\\
Jag är bäst på att dansa.\\
Fyr shot, fem shot, spy.
\end{center}

\begin{flushright}
``Det är dags att gå.''\\
Tre timmar senare dock,\\
är det dags att gå.
\end{flushright}

\begin{flushleft}
Oändlig hunger.\\
Jag skådar en snägäre.\\
``Ge mig en lihis.''
\end{flushleft}

\begin{center}
Äntligen hemma.\\
Jag törstar, öppnar kylen,\\
ser en gatorade.
\end{center}

\vspace{0.8cm}

\textbf{Bok 2: Att vakna}

\begin{flushleft}
Jag drömde om Fymm\\
och vaknade till Kvant. Mek.\\
Vill ej leva mer.
\end{flushleft}

\begin{center}
Ack, vilken fiilis.\\
Ack, var äro ämbaret?\\
Ack, det är för sent.
\end{center}

\begin{flushright}
Ett steg från sängen.\\
Men hela världen snurrar.\\
Lägger mig igen.
\end{flushright}

\begin{flushleft}
Mitt huvve bultar.\\
Det finns ingen burana.\\
Är detta slutet?
\end{flushleft}

\begin{center}
Torr som sahara.\\
Långsam som en sköldpadda.\\
Sätter på Netflix.
\end{center}

\begin{flushright}
Tar fram snapsboken,\\
Läser allas små skrifter.\\
Så många kukar.
\end{flushright}

\begin{flushleft}
Vad strupen törstar.\\
Var äro min gatorade?\\
Nån har druckit den.
\end{flushleft}

\begin{center}
Kokade kaffe.\\
Njuter av svarta guldet.\\
Måste på vessa.
\end{center}

\begin{flushright}
Det värsta är slut.\\
Min mage börjar kurra.\\
Ser en halv lihis.
\end{flushright}

\begin{flushleft}
Tröttheten stiger.\\
Inspirationen lider.\\
Kan ej skriva mer.
\end{flushleft}

\begin{flushright}
\emph{Emil L. - söndrig poet och allmän yrboll}
\end{flushright}

\vspace{1.6cm}
\picture[]{bilder/phdf.jpg}

\end{artikel}


%----------------------------------------------------------------------
%	This command prints the publisher information on one line with
%	all fields separated by 'bullets'.
%----------------------------------------------------------------------
\lastpage{0.9\textwidth}{bilder/RELEX_A5.pdf}

\end{document}
